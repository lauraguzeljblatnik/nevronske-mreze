\documentclass[mat1]{fmfdelo}
% \documentclass[fin1]{fmfdelo}
% \documentclass[isrm1]{fmfdelo}
% \documentclass[mat2]{fmfdelo}
% \documentclass[fin2]{fmfdelo}
% \documentclass[isrm2]{fmfdelo}

% naslednje ukaze ustrezno napolnite
\avtor{Laura Guzelj Blatnik}

\naslov{Nevronske mreže z vzvratnim razširjanjem napak v funkcijskem programskem jeziku}
\title{Angleški prevod slovenskega naslova dela}

% navedite ime mentorja s polnim nazivom: doc.~dr.~Ime Priimek,
% izr.~prof.~dr.~Ime Priimek, prof.~dr.~Ime Priimek
% uporabite le tisti ukaz/ukaze, ki je/so za vas ustrezni
\mentor{prof.~dr.~Ljupčo Todorovski}
% \mentorica{}
 \somentor{asist.~dr.~Aljaž Osojnik}
% \somentorica{}
% \mentorja{}{}
% \mentorici{}{}

\letnica{2020} % leto diplome

%  V povzetku na kratko opišite vsebinske rezultate dela. Sem ne sodi razlaga organizacije dela --
%  v katerem poglavju/razdelku je kaj, pač pa le opis vsebine.
\povzetek{}

%  Prevod slovenskega povzetka v angleščino.
\abstract{}

% navedite vsaj eno klasifikacijsko oznako --
% dostopne so na www.ams.org/mathscinet/msc/msc2010.html
\klasifikacija{}
\kljucnebesede{} % navedite nekaj ključnih pojmov, ki nastopajo v delu
\keywords{} % angleški prevod ključnih besed

\zapisiMetaPodatke  % poskrbi za metapodatke in veljaven PDF/A-1b standard

% aktivirajte pakete, ki jih potrebujete
% \usepackage{tikz}

% za številske množice uporabite naslednje simbole
\newcommand{\R}{\mathbb R}
\newcommand{\N}{\mathbb N}
\newcommand{\Z}{\mathbb Z}
\newcommand{\C}{\mathbb C}
\newcommand{\Q}{\mathbb Q}

% matematične operatorje deklarirajte kot take, da jih bo Latex pravilno stavil
% \DeclareMathOperator{\conv}{conv}

% vstavite svoje definicije ...
%  \newcommand{}{}

\begin{document}

\section{Uvod}

Človeški možgani so kompleksen organ predvsem zaradi neštrtih funkcij, ki jih opravljajo. In zgolj vprašanje časa je bilo, kdaj bodo znanstveniki skoraj neskončne zmožnosti možganov prenesli v računalništvo. Ko so poskušali idejo uresničiti, so si predvsem želeli strukture, ki se bo - podobno kot možgani - sposobna učiti, odzivati na sprememe in prepoznati neznane situacije. Tako se je v polovici prejšnjega stoletja rodila ideja o umetnih nevronskih mrežah. 

Do danes so umetne nevronske mreže že močno napredovale in v nekaterih lastnostih celo prekašajo možgane. Kljub temu so možgani sposobni masrcičesa, česar računalniki ne bodo nikoli.

Diplomska naloga se poglobi v nevronske mreže z vzvratnim razširjanjem napake. Za konec pa sem nevronsko mrežo tudi implementirala s pomočjo funkcijskega programskega jezika OCaml. Funkcijsko programiranje...



\section{Nevronske mreže}

\subsection{Lastnosti nevronskih mrež}

\subsection{Uporaba nevronskih mrež}

\section{Učenje}
Za učenje nevronskih mrež obstaja veliko različnih pravil. V svoji diplomski nalogi sem se osredotočila na posplošeno pravilo delta oziroma vzvratno razširjanje napake. 
\subsection{Ideja}
Za vzvratnim razširjanjem napake stoji povsem preprosta ideja. Najprej uteži poljubno nastavimo, paziti moramo le, da vseh vrednosti ne nastavimo na $0$.
\subsection{Vzvratno razširjanje napake}
Vzemimo splošen večsojen perceptron z $X_{N_X}$ nevroni v vhodnem sloju in $Y_{N_Y}$ nevroni v izhodnem sloju. Nevronska mreža naj sestoji iz $m$ skritih slojev, $m>0$, vsak skriti sloj pa naj vsabuje $N_k$ nevronov, kjer velja $1\leq k \leq m$. Nevrone v skritih slojih označimo sledeče: $H_{ij}$, kjer število $i$ ponazarja sloj v katerem se nevron nahaja, število $j$ pa zaporedno številko nevrona v tem sloju. 

\section{Funkcijski programski jezik OCaml}

\section{Praktični del}

\section{Zaključek}

\section*{Slovar strokovnih izrazov}

\geslo{}{}
\geslo{}{}

% seznam uporabljene literature
\begin{thebibliography}{99}

\bibitem{https://www.researchgate.net/publication/266396438_A_Gentle_Introduction_to_Backpropagation}

%\bibitem{}

\end{thebibliography}

\end{document}

